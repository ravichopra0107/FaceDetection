\chapter{Introduction}
\label{introchap}

This chapter presents an overview about the context as a part of project being developed in section 1.1. In section 1.2 the problems and
motivations are presented. Next, in section 1.3, due to the presented motivations, a
research work flow is introduced step by step. Finally, in section 1.4, the research
objectives are presented.

\section{Context}
This project is a part of B.Tech. Computer Science Engineering curriculum for the third semester. The objective is to develop and experiment with solutions to automate and dynamise the billing system in the mess areas of the institute. This project will replace the current billing system (semester-based) with a dynamic system that bills students on a per meal basis. The aim is to use a face recognition module to achieve the above-stated objectives.


\section{Problem/Motivation}
\textbf{Project Motivation} : The problem with the existing billing system is that it bills all the students the same even if they skip some meals due to some reasons. Moreover, there is no way for our mess workers to predict how much food to prepare, which leads to a lot of food wastage. These reasons motivate me to develop a system that bills students dynamically and generates data for mess workers. This data will help them to estimate the amount of food to prepare to minimise food wastage.\\
\textbf{Face recognition problems}: Most existing face recognition systems suffer from problems like face orientation in the image, varying brightness, noise, etc. Some models also suffer from the problem of overfitting due to the non-linearity of the cosine function. (More details are discussed in further sections).\\
\textbf{Web interface features} : This project includes a web interface for students to view and pay their bills on monthly bills. Every student will be notified as a text message when he/she enters the mess area. A proper database for each student will be maintained where students' personal information will be stored in encrypted form.\\
\textbf{Design and UX} : Various aspects of design and User Experience have been kept in mind during the development of the project. Till now, a design has been made using Canva, a design software.

\section{Objectives}
\textbf{Face recognition module} : The project's objective is to develop a robust face recognition system that can work in a mess area condition. The model must be robust to changes in brightness, orientation and other factors. \\
\textbf{Web application} : The other aim is to provide a web interface for students to interact with the system. It will include a payment gateway developed using one of the APIs available in the market.\\
\textbf{Deployment} : Optionally, we are thinking to deploy the face recognition model on Raspberry Pi. This objective is just a speculation that we try to achieve.


\section{Work flow}
According to the objectives, the report will describe the work flow as below: \\
\textbf{Step 1} Developing/Improving an existing model for face recognition that is not sensitive to changes to brightness and orientation. \\
\textbf{Step 2} Developing a web interface for students to interact with system. \\
\textbf{Step 3} Using requests module in python to help the model interact with web application's server side. \\
\textbf{Step 4} Using Twilio API to send notification messages to students. \\
\textbf{Step 5} Using Payments API to allow students clear their dues.\\
\textbf{Step 6(Optional)} Deployment on Raspberry Pi.
