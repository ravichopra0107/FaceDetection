\chapter{Introduction}
\label{introchap}

This chapter presents an overview of the context as a part of the project developed in
section 1.1. Section 1.2 introduces the objectives of the project. Next, section 1.3 presents the implementation workflow step by step. Finally, in section 1.4, the result of the research carried out is briefly introduced.

\section{Context}
This project is a part of B.Tech. Computer Science Engineering curriculum for the third
semester. The objective is to develop and experiment with solutions to automate and dynamize
the billing system in the mess areas of the institute. This project aims to replace the current billing
system (semester-based) with a dynamic system that bills students per meal. We used a Face verification system to realize the above-stated objectives.

\section{Implementation workflow}
The workflow followed during the implementation is as follows:\\
\textbf{Step 1:} Data collection for Machine learning model\\
\textbf{Step 2:} Prepare an augmentation pipeline\\
\textbf{Step 3:} Train model using ArcFace loss function without modifications\\
\textbf{Step 4:} Record results\\
\textbf{Step 5:} Train model with modified loss function\\
\textbf{Step 6:} Record results\\
\textbf{Step 7:} Build API for system using Node.JS, Express.Js, Twilio API, Stripe API\\
\textbf{Step 8:} Test and document API using Postman\\
\textbf{Step 9:} Build UI for web application using React.JS\\
\textbf{Step 10:} Integrate Frontend and Backend\\
\textbf{Step 11:} Test, and document the final system.

\section{Objectives}
\textbf{Face recognition module:} The project’s objective is to develop a robust face recognition
system that can work in a mess area condition. The model is robust to minor changes in brightness,
orientation and other factors.
\textbf{Web application} : The other aim is to provide a web interface for students to interact with the
system. It will include a payment gateway developed using Stripe API. Twilio API is used as a notifier for students and to provide authentication using OTP(one time password).

\section{Research results}
There is a slight increase in training and test accuracy after modifications. Test accuracy is also marginally better. The best part of the results is that the training time gets drastically reduced after modifications. Detailed results have been discussed further.